
\documentclass[onecolumn, 12pt, twoside, openright, dvipdfm]{book}
\usepackage{fontspec}

\defaultfontfeatures{Scale=MatchLowercase} 
%\setmainfont[Mapping=tex-text]{Cochin} 
%\setmainfont[Mapping=tex-text]{Palatino}
%\setmainfont[Mapping=tex-text]{Baskerville}
\setmainfont[Mapping=tex-text]{Perpetua}
\setsansfont[Mapping=tex-text]{Geneva} 
\setmonofont{Monaco} 

\usepackage{fancyhdr}
\usepackage{makeidx}
\usepackage{fancyvrb}
\makeindex

\usepackage[dvipdfm,CJKbookmarks,bookmarks=true,bookmarksopen=true]{hyperref}
\hypersetup{
    pdftitle={},
    pdfauthor={Abdulaziz Ghuloum},
    pdfkeywords={},
    bookmarksnumbered=true,
    %pagebackref=true,
    breaklinks=true,
%    pdfview=FitH,       % Or try pdfstartview={FitV}, This lead to uncorrect bookmarks
    urlcolor=cyan,
    colorlinks=true,
    citecolor=blue,          %citeref's color
    linkcolor=blue,
}

\usepackage{rotating}
\usepackage{multicol,ragged2e}


\renewcommand{\sectionmark}[1]{\markright{\thesection\ #1}}
\fancyhf{} 
\fancyhead[LE,RO]{\bfseries\thepage}
\fancyhead[LO]{\bfseries\rightmark}
\fancyhead[RE]{\bfseries\leftmark}
\renewcommand{\headrulewidth}{0pt} 
\renewcommand{\footrulewidth}{0pt} 
\renewcommand{\headheight}{15pt} 

\newcommand{\coderefpage}[1]{figure~\ref{#1}, p.~\pageref{#1}}
\newcommand{\figrefpage}[1]{figure~\ref{#1}, p.~\pageref{#1}}
\newcommand{\defref}[1]{\texttt{#1}~(page~\pageref{#1})}

\newcommand{\coderef}[1]{figure~\ref{#1}}
\newcommand{\figref}[1]{figure~\ref{#1}}

\newcommand{\secref}[1]{section~\ref{#1}}
\newcommand{\rnrs}[1]{R$^{\mathrm{#1}}$RS}

\newcommand{\BoxedText}[2]{
  \vspace{.05in}
  \begin{center}
    \begin{tabular}{|p{4.6in}|} {\large \emph{#1}} #2 \end{tabular}
  \end{center}
  \vspace{.05in}
}

\newcommand{\testfile}[2]{\texttt{tests-{\ref{#1}}-{#2}.ss}}

\newcommand{\idxtt}[1]{\index{#1 @ \texttt{#1}}\texttt{#1}}

\newenvironment{Language}
{
  \begin{center}
    \begin{tabular}{|p{4.5in}|} {\large \emph{Input Language:}}
      \begin{center}
        \begin{tabular}{@{}r@{ }c@{ }l@{}}
}{
        \end{tabular}
      \end{center}
    \end{tabular}
  \end{center}
}


\DefineVerbatimEnvironment{CodeInline}{Verbatim}
  {gobble=0,  xleftmargin=2em, xrightmargin=0em,
   %numbers=left, numbersep=2mm,
   frame=lines ,framerule=1pt} 


    
\DefineVerbatimEnvironment{CodeInlineIdx}{Verbatim}
  {gobble=0,  xleftmargin=3em, xrightmargin=0em,
   numbers=left, numbersep=1ex,
   frame=lines ,framerule=1pt} 

\newcommand{\idxlabeldefun}[5]{
\vspace{1ex}
\rule{\textwidth}{2pt}
{\index{#1@\texttt{#2}}\label{#3}{\Large\texttt{#4}}\hfill\textbf{#5}}\\}


\newcommand{\idxdefun}[3]{\idxlabeldefun{#1}{#2}{#1}{#2}{#3}}


\newcommand{\defun}[2]{\idxdefun{#1}{#1}{#2}}

\begin{document}
\frontmatter

\VerbatimFootnotes

\title{Ikarus Scheme User's Guide}

\author{Abdulaziz Ghuloum}

\pagestyle{empty}

\mbox{}
\vspace{3in}

{
\fontsize{66}{66}
\fontspec{Hoefler Text Italic}
% \fontspec{Palatino}
% \rnrs{6} Libraries
% {
% \fontsize{36}{36}
% \fontspec{Palatino}
%  and syntax-case system
% }
\begin{center}
Ikarus Scheme User's Guide
\end{center} 
}
\noindent
\rule{\textwidth}{6pt} 
{\fontsize{18}{18}
\fontspec{Hoefler Text Italic}
\hfill{}
% Quick Start\\
% \rnrs{6} Crash Course\\
% Ikarus
(Preliminary Document)
\hfill Version~0.0.1+
}
\vfill
{
\fontsize{24}{24}
\fontspec{Hoefler Text Italic}
\hfill{} Abdulaziz Ghuloum
}

{
\fontsize{18}{18}
\fontspec{Hoefler Text Italic}
\hfill{} \today
}

\newpage
\mbox{}
\vfill{}
\noindent
Ikarus Scheme User's Guide\\
Copyright \copyright{} 2007,  Abdulaziz Ghuloum\\

\noindent
Permission is granted to copy, distribute and/or modify this
document under the terms of the GNU Free Documentation License,
Version 1.2 published by the Free Software
Foundation; with no Invariant Sections, the Front-Cover Texts
being \emph{``Ikarus Scheme User's Guide''}, and
no Back-Cover Texts.  A copy of the license is included in the
section entitled ``GNU Free Documentation License''.

\newpage

\pagestyle{fancy}
\tableofcontents

\newpage

\mainmatter
\setlength{\parindent}{0pt} 
\setlength{\parskip}{2.0ex plus 0ex minus 0ex}
\chapter{Getting Started}
\section{Introduction}

Ikarus Scheme is an implementation of the Scheme programming
language.  The preliminary release of Ikarus implements the majority
of the features found in the current standard, the
Revised$^\mathrm{6}$ report on the algorithmic language
Scheme\cite{r6rs} including full \rnrs{6} library and script syntax,
syntax-case, unicode strings, bytevectors, user-defined record
types, exception handling, conditions, and enumerations.  Over 80\%
of the \rnrs{6} procedures and keywords are currently implemented
and subsequent releases will proceed towards brining Ikarus to full
\rnrs{6} conformance.

The main purpose behind releasing Ikarus early is to give Scheme
programmers the opportunity to experiment with the various new
features that were newly introduced in \rnrs{6}.  The most important
of such features is the ability to structure large programs into
libraries; where each library extends the language through
procedural and syntactic abstractions.  Many useful libraries can be
written using the currently supported set of \rnrs{6} features
including text processing tools, symbolic logic systems,
interpreters and compilers, and many mathematical and scientific
packages.  It is my hope that this release will encourage the
Scheme community to write and to share their most useful \rnrs{6}
libraries.


\newpage

\section{Technology Overview}

Ikarus Scheme provides the programmer with many advantages:

\textbf{Optimizing code generator:}  The compiler's backend employs
state of the art technologies in code generation that produce fast
efficient machine code.  When developing computationally intensive
programs, one is not constrained by using a slow interpreter.

\textbf{Fast incremental compilation:}  Every library and script is
quickly compiled to native machine code.  When developing large
software, one is not constrained by how slow the batch compiler
runs.

\textbf{Robust and fine-tuned standard libraries:}  The standard
libraries are written such that they perform as much error checking
as required to provide a safe and fast runtime environment. 

\textbf{Multi-generational garbage collector:} The
BiBOP\cite{dybvig:sm} based garbage collector used in Ikarus allows
the runtime system to expand its memory footprint as needed.  The
entire 32-bit virtual address space could be used and unneeded
memory is released back to the operating system.

\textbf{Supports many operating systems:} Ikarus runs on the most
popular and widely used operating systems for servers and personal
computers.  The supported systems include Mac~OS~X,
GNU/Linux, FreeBSD, NetBSD, and Microsoft Windows.


\section{System Requirements}

\subsection{Hardware}

Ikarus Scheme runs on the IA-32 (\emph{x86}) architecture
supporting SSE2 extensions.  This includes the Athlon 64,
Sempron 64, and Turion 64 processors from AMD and the Pentium 4, Xeon,
Celeron, Pentium M, Core, and Core2 processors from Intel.  The
system does not run on Intel Pentium III or earlier
processors.

The Ikarus compiler generates SSE2 instructions to handle Scheme's
IEEE floating point representation (\emph{flonums}) for inexact
numbers. 

\subsection{Operating Systems}

Ikarus is tested under the following operating systems:

\begin{itemize}
\item Mac OS X version 10.4.
\item Linux 2.6.18 (Debian, Fedora, Gentoo, and Ubuntu).
\item FreeBSD version 6.2.
\item NetBSD version 3.1.
\item Microsoft Windows XP (using Cygwin 1.5.24).
\end{itemize}

\subsection{Additional Software}

\begin{itemize}
\item\textbf{GMP:} Ikarus uses the GNU Multiple Precision Arithmetic
Library (GMP) for some bignum arithmetic operations.  To build
Ikarus from scratch, GMP version 4.2 or better must be installed
along with the required header files.  Pre-built GMP packages  are
available for most operating systems.  Alternatively, GMP can be
downloaded from \\
\url{http://gmplib.org/}.

\item\textbf{GCC:} The GNU C Compiler is required to build the Ikarus
executable (e.g. the garbage collector, loader, and OS-related
runtime).  GCC versions 4.1 and 4.2 were successfully used to build
Ikarus.

\item\textbf{Autoconf and Automake:} The GNU Autoconf (version 2.61)
and GNU Automake (version 1.10) tools are required if one
wishes to modify the Ikarus source base.  They are not
required to build the official release of Ikarus.

\item\textbf{XeLaTeX:}  The XeLaTeX typesetting system is required
for building the documentation.  XeLaTeX (and XeTeX) is an
implementation of the LaTeX (and TeX) typesetting system.

\end{itemize}

\BoxedText{Note:}{Ikarus runs in 32-bit mode only.  
To run it in 64-bit environments, you will have to obtain the 32-bit
version of GMP, or compile it yourself after adding \texttt{ABI=32}
to its configuration options.}


\section{Installation}

If you are familiar with installing Unix software on your system,
then all you need to know is that Ikarus uses the standard
installation method found in most other Unix software.  Simply run
the following commands from the shell:
\begin{verbatim}
  $ tar -zxf ikarus-n.n.n.tar.gz
  $ cd ikarus-n.n.n
  $ ./configure [--prefix=path] [CFLAGS=-I/dir] [LDFLAGS=-L/dir]
  $ make
  $ make install
  $
\end{verbatim}

The rest of this section describes the build process in more
details.  It is targeted to users who are unfamiliar with steps
mentioned above.

\subsection{Installation Details}

\begin{enumerate}

\item Download the Ikarus source distribution.  The source is
distributed as a \texttt{gzip}-compressed \texttt{tar} file
(\texttt{ikarus-n.n.n.tar.gz} where \texttt{n.n.n} is a 3-digit
number indicating the current revision).  The latest revision can be
downloaded from the following URL:\\
\url{http://www.cs.indiana.edu/~aghuloum/ikarus/}

\item Unpack the source distribution package.  From your shell
command, type:
\begin{verbatim}
  $ tar -zxf ikarus-n.n.n.tar.gz
  $
\end{verbatim}
This creates the base directory \texttt{ikarus-n.n.n}.

\item Configure the build system by running the \texttt{configure}
script located in the base directory.  To do this, type the
following commands:
\begin{verbatim}
  $ cd ikarus-n.n.n
  $ ./configure
  checking build system type... i386-apple-darwin8.10.1
  checking host system type... i386-apple-darwin8.10.1
  ...
  configure: creating ./config.status
  config.status: creating Makefile
  config.status: creating src/Makefile
  config.status: creating scheme/Makefile
  config.status: creating doc/Makefile
  config.status: executing depfiles commands
  $
\end{verbatim}

This configures the system to be built then installed in the
system-wide location (binaries are installed in
\texttt{/usr/local/bin}) .  If you wish to install it
in another location (e.g. in your home directory), you can supply
a \texttt{--prefix} location to the \texttt{configure} script as
follows:

\begin{verbatim}
  $ ./configure --prefix=/path/to/installation/location
\end{verbatim}

The \texttt{configure} script will fail if it cannot locate the
location where GMP is installed.  If running \texttt{configure}
fails to locate GMP, you should supply the location in which the GMP
header file, \texttt{gmp.h}, and the GMP library file,
\texttt{libgmp.so}, are installed.  This is done by supplying the
two paths in the \texttt{CFLAGS} and \texttt{LDFLAGS} arguments:

\begin{verbatim}
  $ ./configure CFLAGS=-I/path/to/include LDFLAGS=-L/path/to/lib
\end{verbatim}

\item Build the system by running:
\begin{verbatim}
  $ make
\end{verbatim}
This performs two
tasks.  First, it builds the \texttt{ikarus} executable from the C
files located in the \texttt{src} directory.  It then uses the
\texttt{ikarus} executable and the pre-built
\texttt{ikarus.boot.orig} boot file to rebuild the Scheme boot image
file \texttt{ikarus.boot} from the Scheme sources located in the
\texttt{scheme} directory.

\item Install Ikarus by typing:
\begin{verbatim}
  $ make install
\end{verbatim}
If you are installing Ikarus in a system-wide location, you might
need to have administrator privileges (use the \texttt{sudo} or
\texttt{su} commands).

\item Test that Ikarus runs from the command line.
\begin{verbatim}
  $ ikarus
  Ikarus Scheme version 0.0.2
  Copyright (c) 2006-2007 Abdulaziz Ghuloum

  > 
\end{verbatim}
If you get the prompt, then Ikarus was successfully installed on
your system.  You may need to update the \texttt{PATH} variable in
your environment to contain the directory in which the
\texttt{ikarus} executable was installed.

Do not delete the \texttt{ikarus-n.n.n} directory from which you
configured, built, and installed Ikarus.  It will be needed if you
decide at a later time to uninstall Ikarus.

\end{enumerate}

\subsection{Uninstalling Ikarus}

To uninstall Ikarus, use the following steps:

\begin{verbatim}
  $ cd path/to/ikarus-n.n.n
  $ make uninstall
  $
\end{verbatim}

\newpage

\section{\index{Command-line switches}Command-line Switches}

The \texttt{ikarus} executable recognizes a few command-line
switches that influence how Ikarus starts.

\begin{itemize}

\item \texttt{ikarus -h}

The presence of the \texttt{-h} flag causes \texttt{ikarus} to
display a help message then exits.  The help message summarizes the
command-line switches.  No further action is performed.

\item \texttt{ikarus -b path/to/boot/file.boot}

The \texttt{-b} flag (which requires an extra argument) directs
\texttt{ikarus} to use the specified boot file as the initial system
boot file.  \index{Boot files} The boot file is a binary file that
contains all the code and data of the Scheme system.  In the absence
of \texttt{-b} flag, the executable will use the default boot file.
Running \texttt{ikarus~-h} shows the location where the default boot
file was installed.

The rest of the command-line arguments are recognized by the
standard Scheme run time system.   They are processed after the
boot file is loaded.

\item \texttt{ikarus --r6rs-script script-file-name [arguments ...]}

\index{R6RS Script@\rnrs{6} Script} The \texttt{--r6rs-script} argument
instructs Ikarus that the supplied file is an \rnrs{6} script.  See
Section~\ref{sec:scripts} for a short introduction to writing \rnrs{6}
scripts.  The script file name and any additional optional
\texttt{arguments}
can be obtained by calling the
\idxtt{command-line} procedure.

\begin{verbatim}
  $ cat test.ss 
  (import (rnrs))
  (write (command-line))
  (newline)

  $ ikarus --r6rs-script test.ss hi there
  ("test.ss" "hi" "there")
  $
\end{verbatim}

\item \texttt{ikarus files ... [-- arguments ...]}

The lack of an \texttt{--r6rs-script} argument causes Ikarus to
start in interactive mode.  Each of the \texttt{files} is first
loaded, in the interaction environment.  The interaction environment
initially contains all the bindings exported from the
\texttt{(ikarus)} library (see Chapter~\ref{chapter:ikarus}).  The
optional \texttt{arguments} following the \texttt{--} marker can be
obtained by calling the \texttt{command-line} procedure.  In
interactive mode, the first element of the returned list will be the
string \texttt{"*interactive*"}, corresponding to the script name in
\rnrs{6}-script mode.


\BoxedText{Note:}{The interactive mode is intended for quickly
experimenting with the built-in features.  It is intended neither
for developing applications nor for writing any substantial pieces
of code.  The main reason for this is that the interaction between
\rnrs{6} libraries and the interactive environment is not well
understood.  We hope to achieve better interaction between the two
subsystems in the future.}

\end{itemize}

\section{Executable Scripts}

Scheme scripts can be executed using the
\texttt{ikarus~--r6rs-script~\textit{script-name}} command as
described in the previous section.  For convenience, Ikarus
follows the \rnrs{6} recommendations and installs a wrapper program
called \texttt{scheme-script}.  Typically, a script you write would
start with a \texttt{\#!}\ line that directs your operating system
to the interpreter used to evaluate the script file.  The following
example shows a very simple script that uses the
\texttt{scheme-script} command.

\begin{CodeInline}
#!/usr/bin/env scheme-script

(import (rnrs))
(display "Hello World\n")
\end{CodeInline}

\newpage

\section{Installing Additional Libraries}
FIXME
\subsection{\texttt{IKARUS\_LIBRARY\_PATH}}
\index{ikarus library path@\texttt{IKARUS\_LIBRARY\_PATH}}
FIXME


%\subsection{Mac OS X}
%FIXME
%\subsection{GNU/Linux}
%FIXME
%\subsection{Windows/Cygwin}
%FIXME


\chapter{\rnrs{6} Crash Course}

The major difference between \rnrs{5} and \rnrs{6} is the way
in which programs are loaded and evaluated.  

In \rnrs{5}, Scheme implementations typically start as an
interactive session (often referred to as the REPL, or
read-eval-print-loop).  Inside the interactive session, the user
enters definitions and expressions one at a time using the keyboard.
Files, which also contain definitions and expressions, can be loaded
and reloaded by calling the \texttt{load} procedure.  The
environment in which the interactive session starts often contains
implementation-specific bindings that are not found \rnrs{5} and
users may redefine any of the initial bindings.  The semantics of a
loading a file depends on the state of the environment at the time
the file contents are evaluated.

\index{R6RS Script@\rnrs{6} Script!Import} 
%
\rnrs{6} differs from \rnrs{5} in that it specifies how \emph{whole
programs}, or scripts, are compiled and evaluated.  An \rnrs{6}
script is closed in the sense that all the identifiers found in the
body of the script must either be defined in the script or imported
from a library.  \rnrs{6} also specifies how \emph{libraries} can be
defined and used.  While files in \rnrs{5} are typically
\emph{loaded} imperatively into the top-level environments, \rnrs{6}
libraries are \emph{imported} declaratively in scripts and in other
\rnrs{6} libraries. 

\section{\label{sec:scripts}Writing a simple script}

An \rnrs{6} script is a set of definitions and expressions preceded
by an \texttt{import} form.  The \texttt{import} form specifies
the language (i.e. the variable and keyword bindings) in which the
library body is written.  A very simple example of an \rnrs{6}
script is listed below.

\index{Examples!Hello World}
\begin{CodeInline}
#!/usr/bin/env scheme-script
(import (rnrs))
(display "Hello World!\n")
\end{CodeInline}

The first line imports the \texttt{(rnrs)} library.  All the
bindings exported from the \texttt{(rnrs)} library are made
available to be used within the body of the library.  
The exports of the \texttt{(rnrs)} library include variables
(e.g. \texttt{cons}, \texttt{car}, \texttt{display}, etc.) and
keywords (e.g.  \texttt{define}, \texttt{lambda}, \texttt{quote},
etc.).  The second line displays the string \texttt{Hello World!}
followed by a new line character.

In addition to expressions, such as the call to \texttt{display} in
the previous example, a script may define some variables.  The
script below defines the variable \texttt{greeting} and calls the
procedure bound to it.

\begin{CodeInline}
#!/usr/bin/env scheme-script
(import (rnrs))

(define greeting
  (lambda ()
    (display "Hello World!\n")))

(greeting)
\end{CodeInline}

Additional keywords may be defined within a script.  In the example
below, we define the \texttt{(do-times n exprs ...)} macro that
evaluates the expressions \texttt{exprs} \texttt{n} times.  Running
the script displays \texttt{Hello World} 3 times.
\newpage

\begin{CodeInline}
#!/usr/bin/env scheme-script
(import (rnrs))

(define greeting
  (lambda ()
    (display "Hello World!\n")))

(define-syntax do-times
  (syntax-rules ()
    [(_ n exprs ...)
     (let f ([i n])
       (unless (zero? i)
         exprs ...
         (f (- i 1))))]))

(do-times 3 (greeting))
\end{CodeInline}


\section{Writing simple libraries}

A script is intended to be a small piece of the program---useful
abstractions belong to libraries.  The \texttt{do-times} macro that
was defined in the previous section may be useful in places other
than printing greeting messages.  So, we can create a small library,
\texttt{(iterations)} that contains common iteration forms.  

An \rnrs{6} library form is made of four essential parts: (1) the
library name, (2) the set of identifiers that the library exports,
(3) the set of libraries that the library imports, and (4) the body
of the library.  


The library name can be any non-empty list of identifiers.
\rnrs{6}-defined libraries includes \texttt{(rnrs)},
\texttt{(rnrs~unicode)}, \texttt{(rnrs~bytevectors)}, and so on.

The library exports are a set of identifiers that are made available
to importing libraries.  Every exported identifier must be bound: it
may either be defined in the libraries or imported from another
library.  Library exports include variables, keywords, record names,
condition names.

Library imports are similar to script imports: they specify the set
of libraries whose exports are made visible within the body of the
library.  

\index{Invoke}
The body of a library contains definitions (variable, keyword,
record, condition, etc.) followed by an optional set of expressions.
The expressions are evaluated for side effect when needed.


The \texttt{(iteration)} library may be written as follows:

\begin{CodeInline}
(library (iteration)
  (export do-times)
  (import (rnrs))

  (define-syntax do-times
    (syntax-rules ()
      [(_ n exprs ...)
       (let f ([i n])
         (unless (zero? i)
           exprs ...
           (f (- i 1))))])))
\end{CodeInline}

To use the \texttt{(iteration)} library in our script, we add the
name of the library to the script's \texttt{import} form.  This
makes all of \texttt{(iteration)}'s exported identifiers, e.g.
\texttt{do-times},  visible in the body of the script.

\begin{CodeInline}
#!/usr/bin/env scheme-script
(import (rnrs) (iteration))

(define greeting
  (lambda ()
    (display "Hello World!\n")))

(do-times 3 (greeting))
\end{CodeInline}

\section{\rnrs{6} record types}

\rnrs{6} provides ways for users to define new types, called record
types.  A record is a fixed-size data structure with a unique type
(called a record type).  A record may have any finite number of
fields that hold arbitrary values.  This section briefly describes
what we expect to be the most commonly used features of the record
system.  Full details are in the \rnrs{6} Standard Libraries
document\cite{r6rs:lib}.

\subsection{Defining new record types}

To define a new record type,  use the \texttt{define-record-type}
form.  For example, suppose we want to define a new record type for
describing points, where a point is a data structure that has two
fields to hold the point's $x$ and $y$ coordinates.  The following
definition achieves just that:

\begin{CodeInline}
(define-record-type point
  (fields x y))
\end{CodeInline}

The above use of \texttt{define-record-type} defines the following
procedures automatically for you:
\begin{itemize}
\item The constructor \texttt{make-point} that takes two arguments,
\texttt{x} and \texttt{y} and returns a new record whose type is
point.
\item The predicate \texttt{point?}\ that takes an arbitrary value
and returns \texttt{\#t} if that value is a point, \texttt{\#f}
otherwise. 
\item The accessors \texttt{point-x} and \texttt{point-y} that,
given a record of type point, return the value stored in the
\texttt{x} and \texttt{y} fields.
\end{itemize}

Both the \texttt{x} and \texttt{y} fields of the \texttt{point}
record type are \emph{immutable}, meaning that once a record is
created with specific \texttt{x} and \texttt{y} values, they cannot
be changed later.  If you want the fields to be \emph{mutable}, then
you need to specify that explicitly as in the following example.
\newpage

\begin{CodeInline}
(define-record-type point
  (fields (mutable x) (mutable y)))
\end{CodeInline}

This definition gives us, in addition to the constructor, predicate,
and accessors, two additional procedures:
\begin{itemize}
\item The mutators \texttt{set-point-x!} and \texttt{set-point-y!} that,
given a record of type point, and a new value, sets the value stored in the
\texttt{x} field or \texttt{y} field to the new value.
\end{itemize}


\BoxedText{Note:}{Records in Ikarus have a printable representation
in order to enable debugging programs that use records.  Records are
printed in the \texttt{\#[type-name field-values ...]} notation.
For example, \texttt{(write (make-point 1 2))} produces 
\texttt{\#[point 1 2]}.}

\subsection{Extending existing record types}

A record type may be extended by defining new variants of a record
with additional fields.  In our running example, suppose we want
to define a \texttt{colored-point} record type that, in addition to
being a \texttt{point}, it has an additional field: a \emph{color}.
A simple way of achieving that is by using the following record 
definition:

\begin{CodeInline}
(define-record-type cpoint
  (parent point)
  (fields color))
\end{CodeInline}

Here, the definition of \texttt{cpoint} gives us:
\begin{itemize}
\item A constructor \texttt{make-cpoint} that takes three arguments
(\texttt{x}, \texttt{y}, and \texttt{color} in that order) and returns a
\texttt{cpoint} record.
\item A predicate \texttt{cpoint?}\ that takes a single argument and
determines whether the argument is a \texttt{cpoint} record.
\item An accessor \texttt{cpoint-color} that returns the value of
the \texttt{color} field of a \texttt{cpoint} object.
\end{itemize}

All procedures that are applicable to records of type
\texttt{point} (\texttt{point?}, \texttt{point-x},
\texttt{point-y}) are also applicable to records of type
\texttt{cpoint} since a \texttt{cpoint} is also a \texttt{point}.

\subsection{Specifying custom constructors}

The record type definitions explained so far use the default
constructor that takes as many arguments as there are fields and
returns a new record type with the values of the fields initialized
to the arguments' values.  It is sometimes necessary or convenient
to provide a constructor that performs more than the default
constructor.  For example, we can modify the definition of our
\texttt{point} record in such way that the constructor takes either
no arguments, in which case it would return a point located at the
origin, or two arguments specifying the $x$ and $y$ coordinates.  We
use the \texttt{protocol} keyword for specifying such constructor as
in the following example:

\begin{CodeInline}
(define-record-type point
  (fields x y)
  (protocol
    (lambda (new)
      (case-lambda
        [(x y) (new x y)]
        [()    (new 0 0)]))))
\end{CodeInline}

The protocol here is a procedure that takes a constructor procedure
\texttt{new} (\texttt{new} takes as many arguments as there are
fields.) and returns the desired custom constructor that we want
(The actual constructor will be the value of the
\texttt{case-lambda} expression in the example above).
Now the constructor \texttt{make-point} would either take two
arguments which constructs a \texttt{point} record as before, or no
arguments, in which case \texttt{(new 0 0)} is called to construct a
point at the origin. 

Another reason why one might want to use custom constructors is to
precompute the initial values of some fields based on the values of
other fields.  An example of this case is adding a \texttt{distance}
field to the record type which is computed as
$d = \sqrt{x^2+y^2}$.  The protocol in this case may be defined as:

\begin{CodeInline}
(define-record-type point
  (fields x y distance)
  (protocol
    (lambda (new)
      (lambda (x y)
        (new x y (sqrt (+ (expt x 2) (expt y 2))))))))
\end{CodeInline}

Note that derived record types need not be modified when additional
fields are added to the parent record type.  For example, our
\texttt{cpoint} record type still works unmodified even after we
added the new \texttt{distance} field to the parent.  
Calling \texttt{(point-distance (make-cpoint 3 4 \#xFF0000))}
returns \texttt{5.0} as expected.

\subsection{Custom constructors for derived record types}

Just like how base record types (e.g. \texttt{point} in the running
example) may have a custom constructor, derived record types can
also have custom constructors that do other actions.  Suppose that
you want to construct \texttt{cpoint} records using an optional
color that, if not supplied, defaults to the value 0.  To do so, we
supply a \texttt{protocol} argument to \texttt{define-record-type}.
The only difference here is that the procedure \texttt{new} is a
\emph{curried} constructor.  It first takes as many arguments as the
constructor of the parent record type, and returns a procedure that
takes the initial values of the new fields.  

In our example, the constructor for the \texttt{point} record type
takes two arguments.  \texttt{cpoint} extends \texttt{point} with
one new field.  Therefore, \texttt{new} in the definition below
first takes the arguments for \texttt{point}'s constructor, then
takes the initial color value.  The definition below shows how the
custom constructor may be defined.

\begin{CodeInline}
(define-record-type cpoint
  (parent point) 
  (fields color) 
  (protocol
    (lambda (new) 
      (case-lambda 
        [(x y c) ((new x y) c)]
        [(x y)   ((new x y) 0)]))))
\end{CodeInline}


\section{Exception Handling}

The procedure \texttt{with-exception-handler} allows the programmer
to specify how to handle exceptional situations.  It takes two
procedures as arguments: 
\begin{itemize}
\item An exception handler which is a procedure that take a
single argument, the object that was raised.
\item A body thunk which is a procedure with no arguments whose body
is evaluated with the exception handler installed.
\end{itemize}

In addition to installing exception handlers, \rnrs{6} provides two
ways of raising exceptions: \texttt{raise} and
\texttt{raise-continuable}.  We describe the procedure
\texttt{raise-continuable} 
first since it's the simpler of the two.  
For the code below, assume that \texttt{print} is defined as:
\begin{CodeInline}
(define (print who obj)
  (display who) 
  (display ": ")
  (display obj)
  (newline))
\end{CodeInline}

The first example, below, shows how a simple exception handler is
installed.  Here, the exception handler prints the object it
receives and returns the symbol \texttt{there}.  The call to
\texttt{raise-continuable} calls the exception handler, passing it
the symbol \texttt{here}.  When the handler returns, the returned
value becomes the value of the calls to \texttt{raise-continuable}.

\begin{CodeInline}
(with-exception-handler
  (lambda (obj)                         ;;; prints
    (print "handling" obj)              ;;;   handling: here
    'there)                             ;;;   returned: there
  (lambda ()
    (print "returned" (raise-continuable 'here))))
\end{CodeInline}

Exceptional handlers may nest, and in that case, if an exception is
raised while evaluating an inner handler, the outer handler is
called as the following example illustrates:

\begin{CodeInline}
(with-exception-handler
  (lambda (obj)                         ;;; prints
    (print "outer" obj)                 ;;;   inner: here    
    'outer)                             ;;;   outer: there   
  (lambda ()                            ;;;   returned: outer
    (with-exception-handler
      (lambda (obj)
        (print "inner" obj)
        (raise-continuable 'there))
      (lambda ()
        (print "returned" (raise-continuable 'here))))))
\end{CodeInline}

In short, \texttt{with-exception-handler} binds an exception handler
within the dynamic context of evaluating the thunk, and
\texttt{raise-continuable} calls it.

The procedure \texttt{raise} is similar to
\texttt{raise-continuable} except that if the handler returns, a new
exception is raised, calling the next handler in sequence until the
list of handlers is exhausted.

\begin{CodeInline}
(call/cc                          ;;; prints
  (lambda (escape)                ;;;   inner: here
    (with-exception-handler       ;;;   outer: #[condition ---]
      (lambda (obj)               ;;; returns
        (print "outer" obj)       ;;;   12
        (escape 12))
      (lambda ()                            
        (with-exception-handler            
          (lambda (obj)
            (print "inner" obj)
            'there)
          (lambda ()
            (print "returned" (raise 'here))))))))
\end{CodeInline}

Here, the call to \texttt{raise} calls the inner exception handler,
which returns, causing \texttt{raise} to re-raise a non-continuable 
exception to the outer exception handler.  The outer exception
handler then calls the escape continuation.

The following procedure provides a useful example of using the
exception handling mechanism.  Consider a simple definition of the
procedure \texttt{configuration-option} which returns the value
associated with a key where the key/value pairs are stored in an
association list in a configuration file.

\begin{CodeInline}
(define (configuration-option filename key)
  (cdr (assq key (call-with-input-file filename read))))
\end{CodeInline}

Possible things may go wrong with calling
\texttt{configuration-option} including errors opening the file,
errors reading from the file (file may be corrupt), error in
\texttt{assq} since what's read may not be an association list, and
error in \texttt{cdr} since the key may not be in the association
list.  Handling all error possibilities is tedious and error prone.  
Exceptions provide a clean way of solving the problem.  Instead of
guarding against all possible errors, we install a handler that
suppresses all errors and returns a default value if things go
wrong.  Error handling for \texttt{configuration-option} may be
added as follows:

\begin{CodeInline}
(define (configuration-option filename key default)
  (define (getopt)
    (cdr (assq key (call-with-input-file filename read))))
  (call/cc
    (lambda (k)
      (with-exception-handler 
        (lambda (_) (k default))
        getopt))))
\end{CodeInline}


\chapter{\label{chapter:ikarus}The \texttt{(ikarus)} library}

In addition to the libraries listed in the \rnrs{6} standard, Ikarus
contains the \texttt{(ikarus)} library which provides additional
useful features.  The \texttt{(ikarus)} library is a composite
library---it exports a superset of all the supported bindings of
\rnrs{6}.  While not all of the exports of \texttt{(ikarus)} are
documented at this time, this chapter attempts to describe a few of
these useful extensions.  

\idxlabeldefun{\#"!ikarus}{\#"!ikarus}{shebang}{\#!ikarus}{comment}
%\defun{\#"!ikarus}{comment}
%{\index{\#"!ikarus@\texttt{\#"!ikarus}}\label{shebang}{\Large\texttt{\#!ikarus}}
%\hfill \textbf{comment}}

\newpage
\section{Parameters}

Parameters in Ikarus\footnote{Parameters are found in many Scheme
implementations such as Chez Scheme and MzScheme.} are intended for
customizing the behavior of a procedure during the dynamic execution
of some piece of code.  Parameters are first class entities
(represented as procedures) that hold the parameter value.  A
parameter procedure accepts either zero or one argument.  If given
no arguments, it returns the current value of the parameter.  If
given a single argument, it must set the state to the value of the
argument.  Parameters replace the older concept of using starred
\texttt{*global*} customization variables.  For example, instead of
writing:
\begin{verbatim}
   (define *screen-width* 72)
\end{verbatim}
and then mutate the variable \texttt{*screen-width*} with
\texttt{set!}, we could wrap \texttt{*screen-width*} with a 
\texttt{screen-width} parameter as follows:
\begin{verbatim}
   (define *screen-width* 72)
   (define screen-width 
     (case-lambda
       [()  *screen-width*]
       [(x) (set! *screen-width* x)]))
\end{verbatim}

The value of \texttt{screen-width} can now be passed as argument,
returned as a value, and exported from libraries.

\defun{make-parameter}{procedure}
\texttt{
(make-parameter x)\\
(make-parameter x f)
}

As parameters are common in Ikarus, the procedure
\texttt{make-parameter} is defined to model common usage pattern of
parameter construction. 

\paragraph{\texttt{(make-parameter x)}} constructs a parameter
with \texttt{x} as the initial value.  For example, the code above
could be written succinctly as:
\begin{verbatim}
   (define screen-width (make-parameter 72))
\end{verbatim}

\paragraph{\texttt{(make-parameter x f)}} constructs a parameter
which filters the assigned values through the procedure \texttt{f}.
The initial value of the parameter is the result of calling
\texttt{(f~x)}.  Typical used of the filter procedure include
checking some constraints on the passed argument or converting it to
a different data type.  The \texttt{screen-width} parameter may be
constructed more robustly as:
\begin{verbatim}
   (define screen-width 
     (make-parameter 72
       (lambda (w)
         (assert (and (integer? w) (exact? w)))
         (max w 1))))
\end{verbatim}
This definition ensures, through \texttt{assert}, that the argument
passed is an exact integer.  It also ensures, through \texttt{max}
that the assigned value is always positive.


\defun{parameterize}{syntax}
\texttt{(parameterize ([lhs* rhs*] ...) body body* ...)}

Parameters can be assigned to by simply calling the parameter
procedure with a single argument.  The \texttt{parameterize} syntax
is used to set the value of a parameter within the dynamic extent of
the \texttt{body~body*~...} expressions.  

The \texttt{lhs* ...} are expressions, each of which must evaluate
to a parameter.  Such parameters are not necessarily constructed by
\texttt{make-parameter}---any procedure that follows the parameters
protocol works.  

The advantage of using \texttt{parameterize} over explicitly
assigning to parameters (same argument applies to global variables)
is that you're guaranteed that whenever control exits the body of a
\texttt{parameterize} expression, the value of the parameter is
reset back to what it was before the body expressions were entered.
This is true even in the presence of \texttt{call/cc}, errors, and
exceptions.

The following example shows how to set the text property of a
terminal window.  The parameter \texttt{terminal-property} sends an
ANSI escape sequence to the terminal whenever the parameter value is
changed.  The use of \texttt{terminal-property} within
\texttt{parameterize} changes the property before
\texttt{(display~"RED!")} is called and resets it back to normal
when the body returns.

\begin{CodeInline}
(define terminal-property
  (make-parameter "0"
    (lambda (x)
      (display "\x1b;[")
      (display x)
      (display "m")
      x)))

(display "Normal and ")
(parameterize ([terminal-property "41;37"])
  (display "RED!"))
(newline)
\end{CodeInline}



\newpage
\section{Local Modules}

This section is not documented yet.
Please refer to Section~10.5 of Chez Scheme
User's Guide~\cite{csug7}, Chapter~3 of Oscar Waddel's Ph.D
Thesis~\cite{waddell-thesis}, and its POPL99
paper~\cite{waddell-extending} for details on using the
\texttt{module} and \texttt{import} keywords.  Ikarus's internal
module system is similar in spirit to that of Chez Scheme.


\defun{module}{syntax}
\texttt{(module M definitions ... expressions ...)}\\
\texttt{(module definitions ... expressions ...)}
FIXME

\defun{import}{syntax}
\texttt{(import M)}
FIXME

\newpage
\section{Local Library Imports}
\defun{import}{syntax}
\texttt{(import import-spec* ...)}
FIXME

\newpage

\section{\label{sec:gensyms}Gensyms}

Gensym stands for a \emph{generated symbol}---a fresh symbol that is
generated at run time and is guaranteed to be \emph{not}
\texttt{eq?} to any other symbol present in the system.  Gensyms are
useful in many applications including expanders, compilers, and
interpreters when generating an arbitrary number of unique names is
needed.

Ikarus is similar to Chez Scheme in that the readers (including the
\texttt{read} procedure) and writers (including \texttt{write} and
\texttt{pretty-print}) maintain the read/write invariance on
gensyms.  When a gensym is written to an output port, the system
automatically generates a random unique identifier for the gensym.
When the gensym is read back though the \texttt{\#\{gensym\}} read
syntax, a new gensym is \emph{not} regenerated, but instead, it is
looked up in the global symbol table. 

A gensym's name is composed of two parts: a \emph{pretty} string and
a \emph{unique} string.  The Scheme procedure
\texttt{symbol->string} returns the pretty string of the gensym and
not its unique string.  Gensyms are printed by default as 
\texttt{\#\{pretty-string~unique-string\}}.

\defun{gensym}{procedure}
\texttt{(gensym)}\\
\texttt{(gensym string)}\\
\texttt{(gensym symbol)}

The procedure \texttt{gensym} constructs a new gensym.  If passed no
arguments, it constructs a gensym with no pretty name.  The pretty
name is constructed when and if the pretty name of the resulting
gensym is needed.
See \defref{gensym-prefix} and \defref{gensym-count} for details. 

\begin{verbatim}
   > (gensym)
   #{g0 |y0zf>GlFvcTJE0xw|}
   > (gensym)
   #{g1 |U%X&sF6kX!YC8LW=|}
   > (eq? (gensym) (gensym))
   #f
\end{verbatim}

\texttt{(gensym string)} constructs a new gensym with
\texttt{string} as its pretty name.   Similarly,
\texttt{(gensym~symbol)} constructs a new gensym with the pretty
name of \texttt{symbol}, if it has one, as its pretty name.

\begin{verbatim}
   > (gensym "foo")
   #{foo |>VgOllCM&$dSvRN=|}
   > (gensym 'foo)
   #{foo |!TqQLmtw2hoEYfU>|}
   > (gensym (gensym 'foo))
   #{foo |N2C>5O0>C?OROUBU|}
\end{verbatim}



\defun{gensym?}{procedure}
\texttt{(gensym? x)}

The \texttt{gensym?}\ predicate returns \texttt{\#t} if its argument
is a gensym, and returns \texttt{\#f} otherwise.

\begin{verbatim}
   > (gensym? (gensym))
   #t
   > (gensym? 'foo)
   #f
   > (gensym? 12)
   #f
\end{verbatim}

\defun{gensym->unique-string}{procedure}
\texttt{(gensym->unique-string gensym)}

The \texttt{gensym->unique-string} procedure returns the unique name
associated with the gensym argument.

\begin{verbatim}
   > (gensym->unique-string (gensym))
   "YukrolLMgP?%ElcR"
\end{verbatim}


\idxdefun{gensym syntax}{\#\{gensym\}}{reader syntax}
\texttt{\#\{unique-name\}}
\index{\#\{pretty unique\}@\texttt{\#\{pretty unique\}} reader syntax}
\\
\texttt{\#\{pretty-name unique-name\}}
\index{\#\{unique\}@\texttt{\#\{unique\}} reader syntax}
\\
\texttt{\#:pretty-name}
\index{\#:pretty@\texttt{\#:pretty} reader syntax}

Ikarus's \texttt{read} and \texttt{write} procedures extends the
lexical syntax of Scheme by the ability to read and write gensyms
using one of the three forms listed above.

\texttt{\#\{unique-name\}} constructs, at read time, a gensym whose
unique name is the one specified.   If a gensym with the same unique
name already exists in the system's symbol table, that gensym is
returned.
\begin{verbatim}
   > '#{some-long-name}
   #{g0 |some-long-name|}
   > (gensym? '#{some-long-unique-name})
   #t
   > (eq? '#{another-unique-name} '#{another-unique-name})
   #t
\end{verbatim}

The two-part \texttt{\#\{pretty-name unique-name\}} gensym syntax is
similar to the syntax shown above with the exception that if a new
gensym is constructed (that is, if the gensym did not already exist
in the symbol table), the pretty name of the constructed gensym is
set to \texttt{pretty-name}.

\begin{verbatim}
   > '#{foo unique-identifier} 
   #{foo |unique-identifier|}
   > '#{unique-identifier}
   #{foo |unique-identifier|}
   > '#{bar unique-identifier} 
   #{foo |unique-identifier|}
\end{verbatim}

The \texttt{\#:pretty-name} form constructs, at read time, a gensym
whose pretty name is \texttt{pretty-name} and whose unique name is
fresh.  This form guarantees that the resulting gensym is not
\texttt{eq?}\ to any other symbol in the system.

\begin{verbatim}
   > '#:foo
   #{foo |j=qTGlEwS/Zlp2Dj|}
   > (eq? '#:foo '#:foo)
   #f
\end{verbatim}

\defun{generate-temporaries}{example}
\index{Examples!generate-temporaries@\texttt{generate-temporaries}}

The \texttt{(rnrs syntax-case)} library provides a
\texttt{generate-temporaries} procedure, which takes a syntax object
(representing a list of things) and returns a list of fresh
identifiers.  Using \texttt{gensym}, that procedure can be defined
as follows:

\begin{CodeInline}
(define (generate-temporaries* stx)
  (syntax-case stx ()
    [(x* ...) 
     (map (lambda (x)
            (datum->syntax #'unimportant 
              (gensym
                (if (identifier? x) 
                    (syntax->datum x)
                    't))))
          #'(x* ...))]))
\end{CodeInline}

The above definition works by taking the input \texttt{stx} and
destructuring it into the list of syntax objects \texttt{x*~...}.
The inner procedure maps each \texttt{x} into a new syntax object
(constructed with \texttt{datum->syntax}).  The datum is a gensym,
whose name is the same name as \texttt{x} if \texttt{x} is an
identifier, or the symbol \texttt{t} if \texttt{x} is not an
identifier.  The output of \texttt{generate-temporaries*} generates
names similar to their input counterpart:

\begin{verbatim}
   > (print-gensym #f) 
   > (generate-temporaries* #'(x y z 1 2))
   (#<syntax x> #<syntax y> #<syntax z> #<syntax t> #<syntax t>)
\end{verbatim}

\newpage
\section{Printing}

\defun{pretty-print}{procedure}
\texttt{(pretty-print datum)}\\
\texttt{(pretty-print datum output-port)}

The procedure \texttt{pretty-print} is intended for printing Scheme
data, typically Scheme programs, in a format close to how a Scheme
programmer would write it.  Unlike \texttt{write}, which writes its
input all in one line, \texttt{pretty-print} inserts spaces and new
lines in order to produce more pleasant output.

\begin{verbatim}
  (define fact-code
    '(letrec ([fact (lambda (n) (if (zero? n) 1 (* n (fact (- n 1)))))])
       (fact 5)))

  > (pretty-print fact-code)
  (letrec ((fact
            (lambda (n) (if (zero? n) 1 (* n (fact (- n 1)))))))
    (fact 5))
\end{verbatim}

The second argument to \texttt{pretty-print}, if supplied, must be 
an output port.  If not supplied, the \texttt{current-output-port}
is used.

\BoxedText{Limitations:}{As shown in the output above, the current
implementation of \texttt{pretty-print} does not handle printing of
square brackets properly.}

\defun{pretty-width}{parameter}
\texttt{(pretty-width)}\\
\texttt{(pretty-width n)}

The parameter \texttt{pretty-width} controls the number of
characters after which the \texttt{pretty-print} starts breaking
long lines into multiple lines.   The initial value of
\texttt{pretty-width} is set to 60 characters, which is suitable for most
terminals and printed material.   

\begin{verbatim}
   > (parameterize ([pretty-width 40])
        (pretty-print fact-code))
   (letrec ((fact
             (lambda (n)
               (if (zero? n)
                   1
                   (* n (fact (- n 1)))))))
     (fact 5))
\end{verbatim}

Note that \texttt{pretty-width} does not guarantee that 
the output will not extend beyond the specified number.  Very long
symbols, for examples, cannot be split into multiple lines and may
force the printer to go beyond the value of \texttt{pretty-width}.

\defun{format}{procedure}
\texttt{(format fmt-string args ...)}

The procedure \texttt{format} produces a string formatted according
to the value of \texttt{fmt-string} and the supplied arguments.  The
format string contains markers in which the string
representation of each argument is placed.  The markers include:

\begin{itemize}

\item[\texttt{"\~{}s"}] instructs the formatter to place the next argument
as if the procedure \texttt{write} has printed it.  If the argument
contains a string, the string will be quoted and all quotes and
backslashes in the string will be escaped.  Similarly, characters
will be printed using the \texttt{\#\\x} notation.

\item[\texttt{"\~{}a"}] instructs the formatter to place the next argument
as if the procedure \texttt{display} has printed it.  Strings and
characters are placed as they are in the output.

\item[\texttt{"\~{}b"}] instructs the formatter to convert the next
argument to its binary (base 2) representation.  The argument must be an
exact number.  Note that the \texttt{\#b} numeric prefix is not
produced in the output.

\item[\texttt{"\~{}o"}] is similar to \texttt{"\~{}b"} except that
the number is printed in octal (base 8).

\item[\texttt{"\~{}x"}] is similar to \texttt{"\~{}b"} except that
the number is printed in hexadecimal (base 16).

\item[\texttt{"\~{}d"}] outputs the next argument, which can be an
exact or inexact number in its decimal (base 10) representation.

\item[\texttt{"\~{}\~{}"}] instructs the formatter to place a tilde
character, \texttt{\~{}}, in the output without consuming an
argument.

\end{itemize}

\begin{verbatim}
   > (format "message: ~s, ~s, and ~s" 'symbol "string" #\c)
   "message: symbol, \"string\", and #\\c"

   > (format "message: ~a, ~a, and ~a" 'symbol "string" #\c)
   "message: symbol, string, and c"
\end{verbatim}

\defun{printf}{procedure}
\texttt{(printf fmt-string args ...)}

The procedure \texttt{printf} is similar to \texttt{format} except
that the output is sent to the \texttt{current-output-port} instead
of being collected in a string.

\begin{verbatim}
   > (printf "message: ~s, ~s, and ~s\n" 'symbol "string" #\c)
   message: symbol, "string", and #\c

   > (printf "message: ~a, ~a, and ~a\n" 'symbol "string" #\c)
   message: symbol, string, and c
\end{verbatim}


\defun{fprintf}{procedure}
\texttt{(fprintf output-port fmt-string args ...)}

The procedure \texttt{fprintf} is similar to \texttt{printf} except
that the output port to which the output is sent is specified as the
first argument.

\defun{print-graph}{parameter}
\texttt{(print-graph)} \\
\texttt{(print-graph \#t)}\\
\texttt{(print-graph \#f)}

The \texttt{print-graph} parameter controls how the writers (e.g.
\texttt{pretty-print} and \texttt{write}) handle shared and cyclic
data structures.  In Ikarus, all writers detect cyclic data
structures and they all terminate on all input, cyclic or otherwise.

If the value of \texttt{print-graph} is set to \texttt{\#f} (the
default), then the writers does not attempt to detect shared data
structures.  Any part of the input that is shared is printed as if
no sharing is present.

If the value of \texttt{print-graph} is set to \texttt{\#t}, all
sharing of data structures is marked using the \texttt{\#n=} and
\texttt{\#n\#} notation. 

\begin{verbatim}
   > (parameterize ([print-graph #f])
       (let ([x (list 1 2 3 4)])
         (pretty-print (list x x x))))
   ((1 2 3 4) (1 2 3 4) (1 2 3 4))

   > (parameterize ([print-graph #t])
       (let ([x (list 1 2 3 4)])
         (pretty-print (list x x x))))
   (#0=(1 2 3 4) #0# #0#)

   > (parameterize ([print-graph #f])
       (let ([x (list 1 2)])
         (let ([y (list x x x x)])
           (set-car! (last-pair y) y)
           (pretty-print (list y y)))))
   (#0=((1 2) (1 2) (1 2) #0#) #0#)

   > (parameterize ([print-graph #t])
       (let ([x (list 1 2)])
         (let ([y (list x x x x)])
           (set-car! (last-pair y) y)
           (pretty-print (list y y)))))
   (#0=(#1=(1 2) #1# #1# #0#) #0#)
\end{verbatim}


% FIXME 
% \defun{print-unicode}{parameter}
% \texttt{(print-unicode)} \\
% \texttt{(print-unicode \#t)} \\
% \texttt{(print-unicode \#f)}


\defun{print-gensym}{parameter}
\texttt{(print-gensym)}\\
\texttt{(print-gensym \#t)}\\
\texttt{(print-gensym \#f)}\\
\texttt{(print-gensym 'pretty)}

The parameter \texttt{print-gensym} controls how gensyms are printed
by the various writers.  

If the value of \texttt{print-gensym} is \texttt{\#f}, then gensym
syntax is suppressed by the writers and only the gensyms' pretty
names are printed.  If the value of \texttt{print-gensym} is
\texttt{\#t}, then the full \texttt{\#\{pretty~unique\}} syntax is
printed.  Finally, if the value of \texttt{print-gensym} is the
symbol \texttt{pretty}, then gensyms are printed using the
\texttt{\#:pretty} notation.  

\begin{verbatim}
   > (parameterize ([print-gensym #f])
       (pretty-print (list (gensym) (gensym))))
   (g0 g1)

   > (parameterize ([print-gensym #t])
       (pretty-print (list (gensym) (gensym))))
   (#{g2 |KR1M2&CTt1<B0n/m|} #{g3 |FBAb&7NC6&=c82!O|})

   > (parameterize ([print-gensym 'pretty])
       (pretty-print (list (gensym) (gensym))))
   (#:g4 #:g5)
\end{verbatim}

The initial value of \texttt{print-gensym} is \texttt{\#t}.


\defun{gensym-prefix}{parameter}
\texttt{(gensym-prefix)}\\
\texttt{(gensym-prefix string)}

The parameter \texttt{gensym-prefix} specifies the string to be used
as the prefix to generated pretty names.  The default value of
\texttt{gensym-prefix} is the string \texttt{"g"}, which causes
generated strings to have pretty names in the sequence \texttt{g0},
\texttt{g1}, \texttt{g2}, etc. 

\begin{verbatim}
   > (parameterize ([gensym-prefix "var"] [print-gensym #f])
        (pretty-print (list (gensym) (gensym) (gensym))))
   (var0 var1 var2)
\end{verbatim}

Beware that the \texttt{gensym-prefix} controls how pretty names are
generated, and has nothing to do with how \texttt{gensym} constructs
a new gensym.  In particular, notice the difference between the
output in the first example with the output of the examples below:

\begin{verbatim}
   > (pretty-print
       (parameterize ([gensym-prefix "var"] [print-gensym #f])
         (list (gensym) (gensym) (gensym))))
   (g3 g4 g5)

   > (let ([ls (list (gensym) (gensym) (gensym))])
       (parameterize ([gensym-prefix "var"] [print-gensym #f])
         (pretty-print ls)))
   (var5 var6 var7)
\end{verbatim}



\defun{gensym-count}{parameter}
\texttt{(gensym-count)}\\
\texttt{(gensym-count n)}

The parameter \texttt{gensym-count} determines the number 
which is attached to the \texttt{gensym-prefix} when gensyms'
pretty names are generated.  The value of \texttt{gensym-count}
starts at 0 when the system starts and  is incremented every time a
pretty name is generated.  It might be set to any non-negative
integer value. 

\begin{verbatim}
   > (let ([x (gensym)])
       (parameterize ([gensym-count 100] [print-gensym #f])
         (pretty-print (list (gensym) x (gensym)))))
   (g100 g101 g102)
\end{verbatim}

Notice from all the examples so far that pretty names are generated
in the order at which the gensyms are printed, not in the order in
which gensyms were created.

\newpage
\section{Tracing}

\defun{trace-define}{syntax}
\texttt{(trace-define (name . args) body body* ...)}\\
\texttt{(trace-define name expression)}

The \texttt{trace-define} syntax is similar to \texttt{define}
except that the bound value, which must be a procedure, becomes a
traced procedure.  A traced procedure prints its arguments when it
is called and prints its values when it returns.  

\begin{verbatim}
   > (trace-define (fact n)
       (if (zero? n) 1 (* n (fact (- n 1)))))
   > (fact 5)
   |(fact 5)
   | (fact 4)
   | |(fact 3)
   | | (fact 2)
   | | |(fact 1)
   | | | (fact 0)
   | | | 1
   | | |1
   | | 2
   | |6
   | 24
   |120
   120
\end{verbatim}

The tracing facility in Ikarus preserves and shows tail recursion
and distinguishes it from non-tail recursion by showing tail calls
starting at the same line in which their parent was called.  

\begin{verbatim}
   > (trace-define (fact n)
       (trace-define (fact-aux n m)
         (if (zero? n) m (fact-aux (- n 1) (* n m))))
       (fact-aux n 1))
   > (fact 5)
   |(fact 5)
   |(fact-aux 5 1)
   |(fact-aux 4 5)
   |(fact-aux 3 20)
   |(fact-aux 2 60)
   |(fact-aux 1 120)
   |(fact-aux 0 120)
   |120
   120
\end{verbatim}

Moreover, the tracing facility interacts well with continuations and
exceptions.

\begin{verbatim}
   > (call/cc
       (lambda (k) 
         (trace-define (loop n)
           (if (zero? n) 
               (k 'done)
               (+ (loop (- n 1)) 1)))
         (loop 5)))
   |(loop 5)
   | (loop 4)
   | |(loop 3)
   | | (loop 2)
   | | |(loop 1)
   | | | (loop 0)
   done
\end{verbatim}


\defun{trace-lambda}{syntax}
\texttt{(trace-lambda name args body body* ...)}

The \texttt{trace-lambda} macro is similar to \texttt{lambda} except
that the resulting procedure is traced: it prints the arguments it
receives and the results it returns.

\defun{make-traced-procedure}{procedure}
\texttt{(make-traced-procedure name proc)}

The procedure \texttt{make-traced-procedure} takes a name (typically
a symbol) and a procedure.  It returns a procedure similar to
\texttt{proc} except that it traces its arguments and values.

\begin{verbatim}
   > (define (fact n)
       (if (zero? n)
           (lambda (k) (k 1))
           (lambda (k)
             ((fact (- n 1))
              (make-traced-procedure `(k ,n)
                (lambda (v) 
                  (k (* v n))))))))
   > (call/cc
       (lambda (k) 
         ((fact 5) (make-traced-procedure 'K k))))
   |((k 1) 1)
   |((k 2) 1)
   |((k 3) 2)
   |((k 4) 6)
   |((k 5) 24)
   |(K 120)
   120
\end{verbatim}


\newpage
\section{Timing}

This section describes some of Ikarus's timing facilities which may
be useful for benchmarking and performance tuning.

\defun{time}{syntax}
\texttt{(time expression)}

The \texttt{time} macro performs the following:  it evaluates
\texttt{expression}, then prints a summary of the run time
statistics, then returns the values returned by \texttt{expression}.
The run-time summary includes the number of bytes allocated, the
number of garbage collection runs, and the time spent in both the
mutator and the collector.


\begin{verbatim}
   > (let ()                                   ;;; 10 million
       (define ls (time (vector->list (make-vector 10000000))))
       (time (append ls ls))
       (values))
   running stats for (vector->list (make-vector 10000000)):
       3 collections
       672 ms elapsed cpu time, including 547 ms collecting
       674 ms elapsed real time, including 549 ms collecting
       120012328 bytes allocated
   running stats for (append ls ls):
       4 collections
       1536 ms elapsed cpu time, including 1336 ms collecting
       1538 ms elapsed real time, including 1337 ms collecting
       160000040 bytes allocated
\end{verbatim}

Note: The  output listed above is \emph{just a sample} that was
taken at some point on some machine.  The output on your
machine at the time you read this may vary.

\newpage
\defun{time-it}{procedure}
\texttt{(time-it who thunk)}

The procedure \texttt{time-it} takes a datum denoting the name of
the computation and a thunk (i.e. a
procedure with no arguments), invokes the thunk, prints the stats,
and returns the values obtained from invoking the thunk.  
If the value of \texttt{who} is non-false, \texttt{who}
is used when displaying the run-time statistics.  If the value of
\texttt{who} is \texttt{\#f}, then no name for the computation is
displayed.

\begin{verbatim}
   > (time-it "a very fast computation"
       (lambda () (values 1 2 3)))
   running stats for a very fast computation:
       no collections
       0 ms elapsed cpu time, including 0 ms collecting
       0 ms elapsed real time, including 0 ms collecting
       56 bytes allocated
   1
   2
   3

   > (time-it #f (lambda () 12))
   running stats:
       no collections
       0 ms elapsed cpu time, including 0 ms collecting
       0 ms elapsed real time, including 0 ms collecting
       32 bytes allocated
   12
\end{verbatim}

\chapter{Missing Features}

Ikarus does not fully conform to \rnrs{6} yet.  Although it
implements the most immediately useful features of \rnrs{6}
including more than 80\% of \rnrs{6}'s macros and procedures, some
areas are still lacking.  This section summarizes the set of
missing features and procedures.  


\begin{itemize}
\item Numeric tower is complete except for complex numbers.\\
  Consequences: \\
  -- Reader does not recognize complex number notation
    (e.g.~\texttt{5-7i}).\\
  -- Procedures that may construct complex numbers from non-complex
    arguments may signal an error or return an incorrect value
    (for example, \texttt{(sqrt -1)} should \emph{not} be \texttt{+nan.0}).
\item Reader does not recognize \texttt{\#!r6rs} syntax.  It should
be modified to accept both \texttt{\#!r6rs} and \texttt{\#!ikarus}
so that Ikarus-specific reader features (gensym syntax, record
syntax, shared graphs, fasl objects, etc.) can be enabled/disabled as needed.
\item The procedure \texttt{equal?}\ may not terminate on
\texttt{equal?}\ infinite (circular) input.
\item Representation of I/O ports is missing a transcoder field.
\end{itemize}
\newpage
\section{List of missing \rnrs{6} procedures}

The following procedures are missing from \texttt{(rnrs base)}:
\begin{Verbatim}
angle magnitude make-polar make-rectangular
\end{Verbatim}

The following procedures are missing form \texttt{(rnrs bytevectors)}:
\begin{Verbatim}
string->utf16     string->utf32    utf16->string     utf32->string
\end{Verbatim}


The following procedures are missing from \texttt{(rnrs unicode)}:
\begin{Verbatim}
string-downcase   string-foldcase   string-titlecase   string-upcase    
string-normalize-nfc                string-normalize-nfd 
string-normalize-nfkc               string-normalize-nfkd   
\end{Verbatim}


The following procedures are missing from \texttt{(rnrs arithmetic
bitwise)}:
\begin{Verbatim}
bitwise-ior  bitwise-xor   bitwise-if             bitwise-bit-field 
bitwise-copy-bit-field     bitwise-bit-set?        bitwise-copy-bit 
bitwise-reverse-bit-field  bitwise-rotate-bit-field  bitwise-length  
\end{Verbatim}


The following procedures are missing from \texttt{(rnrs arithmetic
fixnum)}:
\begin{Verbatim}
fxreverse-bit-field   fxrotate-bit-field   
\end{Verbatim}


The following procedures are missing from \texttt{(rnrs hashtables)}:
\begin{Verbatim}
make-eqv-hashtable         make-hashtable
hashtable-hash-function    hashtable-equivalence-function 
equal-hash   string-hash   string-ci-hash   symbol-hash
\end{Verbatim}



The following procedures are missing from \texttt{(rnrs io ports)}:
\begin{Verbatim}
call-with-bytevector-output-port     call-with-string-output-port
binary-port?   textual-port?    port-eof?
port-has-port-position?         port-position 
port-has-set-port-position!?    set-port-position!
call-with-port        close-port
get-bytevector-all    get-bytevector-some
get-bytevector-n      get-bytevector-n!
lookahead-char
lookahead-u8
get-string-all        get-string-n  get-string-n!  put-string
get-datum             put-datum
make-custom-binary-input-port         make-custom-binary-input/output-port    
make-custom-binary-output-port        make-custom-textual-input-port    
make-custom-textual-input/output-port make-custom-textual-output-port
open-bytevector-input-port            open-bytevector-output-port 
open-file-input-port   open-file-input/output-port open-file-output-port    
output-port-buffer-mode
transcoded-port       port-transcoderput-bytevector 
standard-error-port   standard-input-port   standard-output-port 
string->bytevector    bytevector->string
\end{Verbatim}




\nocite{ghuloum-implicit}
\nocite{ghuloum-generation}

\backmatter
\appendix
\bibliographystyle{plain}
\bibliography{ikarus-users-guide}
\printindex



\end{document}






